\documentclass[11pt]{article}
\usepackage[a4paper, hmargin={2.8cm, 2.8cm}, vmargin={2.5cm, 2.5cm}]{geometry}
\usepackage{eso-pic} % \AddToShipoutPicture
\usepackage{graphicx} % \includegraphics

%% Change `ku-farve` to `nat-farve` to use SCIENCE's old colors or
%% `natbio-farve` to use SCIENCE's new colors and logo.
\def \ColourPDF {include/ku-farve}

%% Change `ku-en` to `nat-en` to use the `Faculty of Science` header
\def \TitlePDF   {include/ku-en}  % University of Copenhagen

\title{
  \vspace{3cm}
  \Huge{Synopsis} \\
  \Large{Implementation and analysis of 3 Shortest Path Algorithms}
}

\author{
  \Large{Bjarke Kingo Iversen}
  \\ \texttt{bjarkekingo50@gmail.com} \\\\
  \Large{Susanne Truong}
  \\ \texttt{suzze-t@hotmail.com}
}


\date{
    \today
}

\begin{document}


\AddToShipoutPicture*{\put(0,0){\includegraphics*[viewport=0 0 700 600]{\ColourPDF}}}
\AddToShipoutPicture*{\put(0,602){\includegraphics*[viewport=0 600 700 1600]{\ColourPDF}}}

\AddToShipoutPicture*{\put(0,0){\includegraphics*{\TitlePDF}}}

\clearpage\maketitle
\thispagestyle{empty}

\newpage

%% Write your dissertation here.
\section{Problem definition}
To examine and benchmark the complexity of different shortest-path algorithms, e.g. Dijkstra, Bellmann-Ford, and A* Search. We will do this by implementing and comparing them using different data structures and libraries. Finally we will compare the results of these experiments with the theoretical bounds.
\section{Limitation}
The project will not focus on the practical application on real problem instances. Our focus lies on shortest path problems, where there is a single-source and single-destination. We will not examine space requirement, but instead analyze the running time.
\section{Work plan and time schedule}
Our work plan contains six main tasks.
\begin{itemize}
\item Learning how the shortest path algorithms work and what properties they utilize.\\
The product will be background knowledge, so we will be able to dry run the algorithms and implement them in C++ later. The resource needed is relevant literature, such as articles and books. This task is not dependent on other tasks and the expected time needed to solve this task is appromixately 40-50 hours. We may use more time, since relevant information about A* search algorithm is difficult to find.

\item Implementations of three shortest path algorithms written in C++.\\
The outcome is three C++ programs containing the implementation for the Dijkstra's, Bellman-Ford, and A* search algorithm. As for resources, we will need a C++ compiler and relevant software. This task is dependent on knowledge of how the different algorithms work and how they utilize their properties. Knowledge about C++ syntax is required as well. The expected time needed is approximately 100-120 hours.

\item Discussion of which algorithms to use for different cases.\\
The product will be knowledge and argumentation for describing cases and methods using shortest path algorithmics. This task is dependent on knowledge of how the different algorithms work and the running time analysis. We expect that this task will take around 20 hours.

\item Benchmarking and discussion of results.\\
The product is data showing the performance. By comparing the data, we can discuss the pros and cons of the different algorithms. The resources required are our implementations and a framework we can use for the benchmarking. This task will be dependent on us implementing the implementations correctly. The time required is expected to be around 50-60 hours.

\item Running time analysis\\
The product will be knowledge, and arguments herein mathematical reasoning, to be used for later discussion and comparing to benchmarked results. The resources required are knowledge about the algorithms. This task will therefore be dependent on   knowledge of how the different algorithms work. The time required is expected to be around 30-40 hours.

\item Comparison between the results and the theoretical bounds\\
The outcome of this task will be information about the results of our implementations, as well as the theoretical bounds. For this, we will need a program that accurately measures the outcome of our implementation, as well as books or articles about the theoretical values. This task will be dependent on a finished implementation as well with running time analysis, and the expected time to complete this is about 50 hours.

\end{itemize}

\section{Relevant literature}
\begin{itemize}
\item Nils J. Nilsson, \textit{Principles of Artificial Intelligence}, Stanford University, Morgan Kaufmann Publishers, Inc. 1980

\item Cormen, T. H., Leiserson, C. E., Rivest, R. L., Stein, C., \textit{Introduction to Algorithms}, 3rd edition, MIT Press, 2009

\item E. W. Dijkstra, \textit{A Note on Two Problems in Connexion with Graphs}, \\ http://www-m3.ma.tum.de/foswiki/pub/MN0506/WebHome/dijkstra.pdf, 1959

\item Melissa Yan, \textit{Dikjkstra's algorithm (slides)}, \\ http://math.mit.edu/~rothvoss/18.304.3PM/Presentations/1-Melissa.pdf, 2014

\item Amit J Patel, \textit{Introduction to A*}, \\ http://theory.stanford.edu/~amitp/GameProgramming/AStarComparison.html, 2016

\item Nick Bruun, \textit{Easy C++ benchmarking}, \\ https://bruun.co/2012/02/07/easy-cpp-benchmarking, 2012

\end{itemize}


\end{document}

