\documentclass[11pt]{article}
\usepackage[a4paper, hmargin={2.8cm, 2.8cm}, vmargin={2.5cm, 2.5cm}]{geometry}
\usepackage{eso-pic} % \AddToShipoutPicture
\usepackage{graphicx} % \includegraphics

%% Change `ku-farve` to `nat-farve` to use SCIENCE's old colors or
%% `natbio-farve` to use SCIENCE's new colors and logo.
\def \ColourPDF {include/ku-farve}

%% Change `ku-en` to `nat-en` to use the `Faculty of Science` header
\def \TitlePDF   {include/ku-en}  % University of Copenhagen

\title{
  \vspace{3cm}
  \Huge{Some Title} \\
  \Large{More elaborate subtitle}
}

\author{
  \Large{Bjarke Kingo Iversen}
  \\ \texttt{bob@gmail.com} \\\\
  \Large{Susanne Truong}
  \\ \texttt{alice@gmail.com}
}


\date{
    \today
}

\begin{document}


\AddToShipoutPicture*{\put(0,0){\includegraphics*[viewport=0 0 700 600]{\ColourPDF}}}
\AddToShipoutPicture*{\put(0,602){\includegraphics*[viewport=0 600 700 1600]{\ColourPDF}}}

\AddToShipoutPicture*{\put(0,0){\includegraphics*{\TitlePDF}}}

\clearpage\maketitle
\thispagestyle{empty}

\newpage

%% Write your dissertation here.
\section{Abstract}
\newpage
\tableofcontents
\newpage
\section{Introduction}
It is widely known that it can be time-consuming to choose the wrong path, if you want to travel from one city to another. It will most certainly be faster to aboard the train from Chicago to Seattle, rather than going through Los Angeles before heading to Seattle. Choosing the right path will therefore save a lot of time.\\
For computersystems to choose such a path, they'd need a method for calculating such, an algorithm.\\
Different shortest-path algorithms are known to solve such problems theoretically. We'll examine the theory behind these, and analyse the complexity to see whether the time-complexity given by Bellmann-Ford and Dijkstras algorithms holds in real life implementations, by implementing them with both a binary and a fibonacci-heap in C++, thus finally to match them all up against each other.\\
We'll as well analyse and implement the A*-algorithm to see how well dijkstra match up against this on euclidean planar graphs.\\
\section{Problem definition}
To examine and benchmark the complexity of different shortest-path algorithms, e.g. Dijkstra, Bellmann-Ford, and A* Search. We will do this by implementing and comparing them using different data structures and libraries. Finally we will compare the results of these experiments with the theoretical bounds.
\section{Bellmann Ford}
\section{Dijkstra}
\section{A* Search}
\section{Results}
\section{Bibliography}

\end{document}
